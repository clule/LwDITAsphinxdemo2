%% Generated by Sphinx.
\def\sphinxdocclass{report}
\documentclass[letterpaper,10pt,english]{sphinxmanual}
\ifdefined\pdfpxdimen
   \let\sphinxpxdimen\pdfpxdimen\else\newdimen\sphinxpxdimen
\fi \sphinxpxdimen=.75bp\relax
%% turn off hyperref patch of \index as sphinx.xdy xindy module takes care of
%% suitable \hyperpage mark-up, working around hyperref-xindy incompatibility
\PassOptionsToPackage{hyperindex=false}{hyperref}

\PassOptionsToPackage{warn}{textcomp}

\catcode`^^^^00a0\active\protected\def^^^^00a0{\leavevmode\nobreak\ }
\usepackage{cmap}
\usepackage{xeCJK}
\usepackage{amsmath,amssymb,amstext}
\usepackage{polyglossia}
\setmainlanguage{english}



\setmainfont{FreeSerif}[
  Extension      = .otf,
  UprightFont    = *,
  ItalicFont     = *Italic,
  BoldFont       = *Bold,
  BoldItalicFont = *BoldItalic
]
\setsansfont{FreeSans}[
  Extension      = .otf,
  UprightFont    = *,
  ItalicFont     = *Oblique,
  BoldFont       = *Bold,
  BoldItalicFont = *BoldOblique,
]
\setmonofont{FreeMono}[
  Extension      = .otf,
  UprightFont    = *,
  ItalicFont     = *Oblique,
  BoldFont       = *Bold,
  BoldItalicFont = *BoldOblique,
]


\usepackage[Sonny]{fncychap}
\ChNameVar{\Large\normalfont\sffamily}
\ChTitleVar{\Large\normalfont\sffamily}
\usepackage{sphinx}

\fvset{fontsize=\small}
\usepackage{geometry}

% Include hyperref last.
\usepackage{hyperref}
% Fix anchor placement for figures with captions.
\usepackage{hypcap}% it must be loaded after hyperref.
% Set up styles of URL: it should be placed after hyperref.
\urlstyle{same}
\addto\captionsenglish{\renewcommand{\contentsname}{Contents:}}

\usepackage{sphinxmessages}
\setcounter{tocdepth}{2}



\title{sphinxdemo}
\date{2020 年 02 月 01 日}
\release{1.0.0a1}
\author{clule}
\newcommand{\sphinxlogo}{\vbox{}}
\renewcommand{\releasename}{发布}
\makeindex
\begin{document}

\pagestyle{empty}
\sphinxmaketitle
\pagestyle{plain}
\sphinxtableofcontents
\pagestyle{normal}
\phantomsection\label{\detokenize{index::doc}}



\chapter{A Brief Introduction}
\label{\detokenize{chap1:a-brief-introduction}}\label{\detokenize{chap1::doc}}
DITA, 全称达尔文信息分类体系结构(Darwin Information Typing Architecture)是一种文档类型定义(Document Type Definitions,DTD)系统,定义了编写和交付技术信息的约定。
为了辅助技术写作人员归类、组织架构信息,DITA框架本身提供多样的话题种类(Topic),例如concept, task, reference, troubleshooting等等。随着DITA版本的不断更新,元素和标签的种类也越来越多。DITA 1.3版本中完整的元素列表中竟然有超过600个元素,所有元素在一个项目中同时出现的情况少之又少,使得。过多的限定内容导致学习、使用成本的增加,使得这些辅助内容成为了写作人员使用DITA的最大障碍。

轻量级DITA(Light Weight DITA,以下简称LwDITA)的出现正是为了解决DITA本身使用复杂的问题。为了减少写作人员的负担,信息架构师基于DITA 1.2版本定义了一种更小的元素集合,在限定的元素集合内,技术写作人员可以不用顾及那些庞杂、冗余的元素集合,更轻松的使用DITA。此外,LwDITA还支持了XML,HTML5和Markdown之间的映射,从而可以支持技术写作人员跨不同的标记语言进行创作,协作和发布。


\chapter{Why Use LwDITA}
\label{\detokenize{chap2:why-use-lwdita}}\label{\detokenize{chap2::doc}}

\section{适用场景}
\label{\detokenize{chap2:id1}}
LwDITA是一种基于标准的替代品,它适用于:

\sphinxstylestrong{1. 使用DITA 1.3过于复杂的情况:}

DITA 1.3 拥有成熟的结构和多样化的功能,但这种成熟的结构也可能让那些希望采用DITA进行写作的技术人员望而却步。LwDITA在提供精简后DITA解决方案的同时,依旧支持与DITA 1.3 的完全兼容。

\sphinxstylestrong{2. 不使用XML作为创作基础的技术协作人员:}

有些技术写作人员依赖除XML以外的特定格式进行写作(例如Markdown或HTML)。尽管这些格式与XML的表现稍有区别,也可以融入到DITA生态系统当中。为了方便他们的使用,LwDITA是第一个跨格式的DITA版本,支持HTML5和Markdown编辑。

——————————————————————

但LwDITA不是DITA 1.3 的替代品。一直使用DITA进行创作的技术人员可以尝试使用LwDITA,但他们并不是LwDITA的主要受众。LwDITA真正关注的,是那些之前并没有使用DITA,或者不熟悉使用XML进行创作的团队。LwDITA会作为这些用户创作和复用内容的入门工具。

LwDITA本意是作为DITA 1.3 版本的一个子集。


\bigskip\hrule\bigskip



\section{核心}
\label{\detokenize{chap2:id2}}\begin{itemize}
\item {} 
提供精简的DITA使用体验

\item {} 
提供XML,HTML5,Markdown之间的映射。由此,用户可以:
\begin{itemize}
\item {} 
使用熟悉的格式进行内容创作

\item {} 
多种标记语言格式内容的轻松发布

\end{itemize}

\item {} 
??支持LwDITA的新型、低成本应用的出现
这三个方面会在接下来文章中进行详细介绍。

\end{itemize}


\subsection{精简的结构}
\label{\detokenize{chap2:id3}}
简化DITA到最小程度是设计LwDITA的主要目的之一。
LwDITA支持DITA标准核心功能,例如:语法标记(semantic tagging)、话题指向(topic orientation)、内容复用(content reuse)等等。但为了在更多行业领域内适用,采用了较DITA 1.3相比更通用的结构,但由此也限制了本身的元素种类、属性、功能,同时降低了适用难度。
上到会议展示,下到使用者的博客,有时会形容DITA的元素类型多的吓人:DITA 1.3拥有189种元素类型;而相比之下LwDITA48种,其中39种从DITA 1.3中沿袭而来,另外增加了9种多媒体元素种类。


\subsection{多种标记语言的支持}
\label{\detokenize{chap2:id4}}
开发LwDITA的另一个目的是希望讲DITA从XML语法中解放出来。许多刚毕业的软件开发人员对XML构造几乎没有直接的经验,而对Markdown和HTML等其他格式则更加熟悉。如今,技术写作人员采用HTML5进行网络结构化文档创建已经非常普遍;Markdown语言也被许多行业和学术界所接受。LwDITA针对这种情况,增加了对两种标记语言的支持,根据语言的不同,LwDITA分为以下三种:
\begin{itemize}
\item {} 
\sphinxstylestrong{XDITA:} 基于XML的LwDITA

\item {} 
\sphinxstylestrong{HDITA:} 基于HTML-5的LwDITA

\item {} 
\sphinxstylestrong{MDITA:} 基于Markdown的DITA

\end{itemize}

这些创作格式可以增强跨部门之间的合作,例如工程师可以适用Markdown编写,网站营销人员可以使用HTML5编写,技术写作人员和其他熟悉DITA的人可以用XML编写。不同语言编写的问价可以轻松集合成单个文档集合发布,并集成在DITA 1.3当中。

\sphinxstylestrong{未来发展趋势:} 可能会逐步增加DITA与JSON、AsciiDoc、MS Word之间的映射。


\subsection{低成本LwDITA工具的出现}
\label{\detokenize{chap2:lwdita}}
DITA技术委员会希望通过简化DITA,降低企业开发平价DITA工具的成本。

DITA 1.3版本的情况是:过多的元素和功能限制了企业开发基于DITA的写作、发布的新平台系统。结构精简的LwDITA避免了这一情况,为未来可能出现的商业、开源应用做好铺垫。


\bigskip\hrule\bigskip



\section{LwDITA与标准DITA的区别}
\label{\detokenize{chap2:lwditadita}}
以下列举了一些LwDITA与标准DITA之间的显著区别,但再LwDITA规范确定之后可能会有所更改:
\begin{itemize}
\item {} 
不允许混合内容。所有文本内容必须存在于‹p›元素中。比如:++‹li›这是一个列表项‹/li›++ 这一句存在语法错误。要想改正需要增加‹p›标签,例如:++‹li› ‹p›这是一个列表项‹/p› ‹/li›++ 此是为了确保跨内容的统一,和结构的可预测性,使开发样式表和工具来处理内容变得更加容易

\item {} 
没有CALS表元素(‹table›,‹row›,‹entry›等)。举个例子,这意味着用户无法创建具有合并单元格,标题或特定列宽的复杂表格。

\item {} 
没有prolog元信息(所有内容都在‹data›标签中)

\item {} 
没有相关链接

\item {} 
只有突出显示highlight的域中的子集:‹b›,‹i›,‹u›,‹sup›,‹sub›标签可用

\item {} 
只有‹topicmeta›、‹topicref›可用,没有‹topichead›和‹topicgroup›

\item {} 
没有‹topicmeta›和‹topicref›可用,没有‹topichead›和‹topicgroup›标签

\item {} 
地图没有‹title›元素。标题(如需要添加)可以放在‹topicmeta›中的‹navtitle›中

\item {} 
仅‹topicmeta›和‹topicref›可用;没有‹topichead›或‹topicgroup›

\item {} 
属性可以随时启用和关闭。其中,默认只有@props可用,其他属性可以进行自定义添加

\end{itemize}


\subsection{LwDITA和标准DITA的兼容性}
\label{\detokenize{chap2:id5}}
像前文说的那样,LwDITA可以说算是标准DITA的一个子集,因此,大多数LwDITA内容都与标准DITA环境兼容。

此外,在标准地图(map)中允许LwDITA主题和标准DITA主题同时出现。但在LwDITA环境中并不能使用DITA标准版本的地图和属性。


\chapter{LwDITA设计}
\label{\detokenize{chap3:lwdita}}\label{\detokenize{chap3::doc}}

\section{LwDITA的话题组成}
\label{\detokenize{chap3:id1}}
LwDITA通过精筛,保留了以下文档组成部分:
\begin{itemize}
\item {} 
Body

\item {} 
Cross reference

\item {} 
Data

\item {} 
Description

\item {} 
Figure

\item {} 
Footnote

\item {} 
Image and alternate text

\item {} 
In-line formatting: Bold, italics, underline, subscript, superscript

\item {} 
Lists: Definition list, ordered list, unordered list

\item {} 
Note

\item {} 
Paragraph

\item {} 
Phrase

\item {} 
Prolog

\item {} 
Preformatted text

\item {} 
Section

\item {} 
Short description

\item {} 
Table

\item {} 
Title

\item {} 
Topic

\end{itemize}


\section{LwDITA的地图组成}
\label{\detokenize{chap3:id2}}
LwDITA通过精筛,保留了以下地图组成元素:
\begin{itemize}
\item {} 
Data

\item {} 
In-line formatting: Bold, italics, underline, superscript, subscript

\item {} 
Key definition

\item {} 
Link text

\item {} 
Map

\item {} 
Navigation title

\item {} 
Phrase

\item {} 
Topic metadata

\item {} 
Topic reference

\end{itemize}


\section{更加严格的模型限制}
\label{\detokenize{chap3:id3}}
更加严格的模型减少了写作时进行选择时的思考成本,但依赖于一些硬性规定。例如在XDITA和HDITA中,除了description、short description、和title元素之外,所有的文字必须包含在段落标签中,提高信息复用性。在段落中可以出现Bold、Italics、Phrase、Superscript、Subscrip、Underline几种形式。

DITA 1.3中的标签形式:

\begin{sphinxVerbatim}[commandchars=\\\{\}]
\PYG{n+nt}{\PYGZlt{}section}\PYG{n+nt}{\PYGZgt{}}Compatible light bulbs include the following:
\PYG{n+nt}{\PYGZlt{}ul}\PYG{n+nt}{\PYGZgt{}}
\PYG{n+nt}{\PYGZlt{}li}\PYG{n+nt}{\PYGZgt{}}Compact Fluorescent\PYG{n+nt}{\PYGZlt{}/li\PYGZgt{}}
\PYG{n+nt}{\PYGZlt{}li}\PYG{n+nt}{\PYGZgt{}}Light Emitting Diode\PYG{n+nt}{\PYGZlt{}/li\PYGZgt{}}
\PYG{n+nt}{\PYGZlt{}/ul\PYGZgt{}}
\PYG{n+nt}{\PYGZlt{}/section\PYGZgt{}}
\end{sphinxVerbatim}

对比LwDITA中的有效标签形式:

\begin{sphinxVerbatim}[commandchars=\\\{\}]
\PYG{n+nt}{\PYGZlt{}section}\PYG{n+nt}{\PYGZgt{}}
\PYG{n+nt}{\PYGZlt{}p}\PYG{n+nt}{\PYGZgt{}}Compatible light bulbs include the following:\PYG{n+nt}{\PYGZlt{}/p\PYGZgt{}}
\PYG{n+nt}{\PYGZlt{}ul}\PYG{n+nt}{\PYGZgt{}}
\PYG{n+nt}{\PYGZlt{}li}\PYG{n+nt}{\PYGZgt{}}
\PYG{n+nt}{\PYGZlt{}p}\PYG{n+nt}{\PYGZgt{}}Compact Fluorescent\PYG{n+nt}{\PYGZlt{}/p\PYGZgt{}}
\PYG{n+nt}{\PYGZlt{}/li\PYGZgt{}}
\PYG{n+nt}{\PYGZlt{}li}\PYG{n+nt}{\PYGZgt{}}
\PYG{n+nt}{\PYGZlt{}p}\PYG{n+nt}{\PYGZgt{}}Light Emitting Diode\PYG{n+nt}{\PYGZlt{}/p\PYGZgt{}}
\PYG{n+nt}{\PYGZlt{}/li\PYGZgt{}}
\PYG{n+nt}{\PYGZlt{}/ul\PYGZgt{}}
\PYG{n+nt}{\PYGZlt{}/section\PYGZgt{}}
\end{sphinxVerbatim}


\section{多媒体内容}
\label{\detokenize{chap3:id4}}
LwDITA为\sphinxstylestrong{多媒体内容}添加了新的元素类型。 这些元素类型与
HTML5兼容,接下来也会在DITA 1.3标准版中出现。
DITA 1.3规范中将‹object›元素类型设置为
在主题中包含的多媒体内容,与HTML中的‹object›对应。 LwDITA更新了XML到HTML元素类型的对应关系,将原来存在于DITA 1.3中的‹object›和‹param›标签进行细化,整理为以下几个元素类型:
\begin{itemize}
\item {} 
Audio

\item {} 
Autoplay

\item {} 
Controls

\item {} 
Loop

\item {} 
Muted

\item {} 
Poster

\item {} 
Source

\item {} 
Track

\item {} 
Video

\end{itemize}

\sphinxstylestrong{注:} 多媒体元素不能在MDITA中使用。


\chapter{LwDITA的种类}
\label{\detokenize{chap4:lwdita}}\label{\detokenize{chap4::doc}}

\section{XDITA}
\label{\detokenize{chap4:xdita}}
XDITA是基于XML的LwDITA,也是DITA标准版本的一个子集,并额外增加了新媒体元素类型。

\sphinxstylestrong{面向对象:}
XDITA面向不需要使用DITA标准版全部功能的用户设计。

XDITA的++潜在用户++包括:
\begin{itemize}
\item {} 
使用XML编辑器工作,但仅使用少量元素和属性的用户

\item {} 
希望减少开发、维护样式表成本的部门

\item {} 
希望已有的DITA内容兼容Markdown或者HTML5的内容创作者

\sphinxstylestrong{举例}

\end{itemize}

XDITA topic(基于DITA标准版元素类型,新增加多媒体元素,如 ++\textless{} video \textgreater{}++ 标签)

\begin{sphinxVerbatim}[commandchars=\\\{\}]
\PYG{c+cp}{\PYGZlt{}?xml version=\PYGZdq{}1.0\PYGZdq{} encoding=\PYGZdq{}UTF\PYGZhy{}8\PYGZdq{}?\PYGZgt{}}
\PYG{c+cp}{\PYGZlt{}!DOCTYPE topic PUBLIC \PYGZdq{}\PYGZhy{}//OASIS//DTD LIGHTWEIGHT DITA Topic//EN\PYGZdq{} \PYGZdq{}lw\PYGZhy{}topic.dtd\PYGZdq{}\PYGZgt{}}
\PYG{n+nt}{\PYGZlt{}topic} \PYG{n+na}{id=}\PYG{l+s}{\PYGZdq{}install\PYGZhy{}and\PYGZhy{}setup\PYGZdq{}}\PYG{n+nt}{\PYGZgt{}}
\PYG{n+nt}{\PYGZlt{}title}\PYG{n+nt}{\PYGZgt{}}Installing and Setting up Remote Lighting\PYG{n+nt}{\PYGZlt{}/title\PYGZgt{}}
\PYG{n+nt}{\PYGZlt{}shortdesc}\PYG{n+nt}{\PYGZgt{}}Installation of your lighting kit includes installing the light bulbs
into light fixtures, preparing the remote control, and programming lighting groups.
\PYG{n+nt}{\PYGZlt{}/shortdesc\PYGZgt{}}
\PYG{n+nt}{\PYGZlt{}prolog}\PYG{n+nt}{\PYGZgt{}}
\PYG{n+nt}{\PYGZlt{}data} \PYG{n+na}{name=}\PYG{l+s}{\PYGZdq{}author\PYGZdq{}} \PYG{n+na}{value=}\PYG{l+s}{\PYGZdq{}Kevin Lewis\PYGZdq{}}\PYG{n+nt}{/\PYGZgt{}}
\PYG{n+nt}{\PYGZlt{}/prolog\PYGZgt{}}
\PYG{n+nt}{\PYGZlt{}body}\PYG{n+nt}{\PYGZgt{}}
\PYG{n+nt}{\PYGZlt{}section}\PYG{n+nt}{\PYGZgt{}}
\PYG{n+nt}{\PYGZlt{}title}\PYG{n+nt}{\PYGZgt{}}Steps\PYG{n+nt}{\PYGZlt{}/title\PYGZgt{}}
\PYG{n+nt}{\PYGZlt{}ul}\PYG{n+nt}{\PYGZgt{}}
\PYG{n+nt}{\PYGZlt{}li}\PYG{n+nt}{\PYGZgt{}}\PYG{n+nt}{\PYGZlt{}p}\PYG{n+nt}{\PYGZgt{}}Install light bulbs.\PYG{n+nt}{\PYGZlt{}/p\PYGZgt{}}\PYG{n+nt}{\PYGZlt{}/li\PYGZgt{}}
\PYG{n+nt}{\PYGZlt{}li}\PYG{n+nt}{\PYGZgt{}}\PYG{n+nt}{\PYGZlt{}p}\PYG{n+nt}{\PYGZgt{}}Prepare remote control.\PYG{n+nt}{\PYGZlt{}/p\PYGZgt{}}\PYG{n+nt}{\PYGZlt{}/li\PYGZgt{}}
\PYG{n+nt}{\PYGZlt{}li}\PYG{n+nt}{\PYGZgt{}}\PYG{n+nt}{\PYGZlt{}p}\PYG{n+nt}{\PYGZgt{}}Program lighting groups.\PYG{n+nt}{\PYGZlt{}/p\PYGZgt{}}\PYG{n+nt}{\PYGZlt{}/li\PYGZgt{}}
\PYG{n+nt}{\PYGZlt{}/ul\PYGZgt{}}
\PYG{n+nt}{\PYGZlt{}/section\PYGZgt{}}
\PYG{n+nt}{\PYGZlt{}section}\PYG{n+nt}{\PYGZgt{}}
\PYG{n+nt}{\PYGZlt{}title}\PYG{n+nt}{\PYGZgt{}}Example\PYG{n+nt}{\PYGZlt{}/title\PYGZgt{}}
\PYG{n+nt}{\PYGZlt{}p}\PYG{n+nt}{\PYGZgt{}}The following video demonstrates a recommended installation:\PYG{n+nt}{\PYGZlt{}/p\PYGZgt{}}
\PYG{n+nt}{\PYGZlt{}video}\PYG{n+nt}{\PYGZgt{}}
\PYG{n+nt}{\PYGZlt{}video\PYGZhy{}poster} \PYG{n+na}{value=}\PYG{l+s}{\PYGZdq{}remote\PYGZhy{}poster.jpg\PYGZdq{}} \PYG{n+nt}{/\PYGZgt{}}
\PYG{n+nt}{\PYGZlt{}media\PYGZhy{}controls} \PYG{n+nt}{/\PYGZgt{}}
\PYG{n+nt}{\PYGZlt{}media\PYGZhy{}source} \PYG{n+na}{value=}\PYG{l+s}{\PYGZdq{}remote.mp4\PYGZdq{}} \PYG{n+nt}{/\PYGZgt{}}
\PYG{n+nt}{\PYGZlt{}/video\PYGZgt{}}
\PYG{n+nt}{\PYGZlt{}/section\PYGZgt{}}
\PYG{n+nt}{\PYGZlt{}/body\PYGZgt{}}
\PYG{n+nt}{\PYGZlt{}/topic\PYGZgt{}}
\end{sphinxVerbatim}


\section{HDITA}
\label{\detokenize{chap4:hdita}}
HDITA是基于HTML-5的LwDITA,包含定制的属性,其中HTML中一些标签的使用与DITA相互转换。

\sphinxstylestrong{面向对象}
熟悉使用HTML进行结构化内容开发的人员。

HDITA的++潜在用户++包括:
\begin{itemize}
\item {} 
市场文档写作者,不使用XML编辑器,但能贡献基于DITA的产品信息

\item {} 
使用HTML写作工具的软件开发人员

\item {} 
想使用移动端设备编辑内容的博主或内容架构师

\item {} 
只需编写内容但不想要格式转换过程的文档作者

\item {} 
熟悉和偏好HTML5的文档写作者

\sphinxstylestrong{举例}

\end{itemize}

HDITA topic(使用与HTML5相同的元素类型和属性,提高复用性和与DITA的兼容性)

\begin{sphinxVerbatim}[commandchars=\\\{\}]
\PYG{c+cp}{\PYGZlt{}!DOCTYPE html\PYGZgt{}}
\PYG{p}{\PYGZlt{}}\PYG{n+nt}{html}\PYG{p}{\PYGZgt{}}
\PYG{p}{\PYGZlt{}}\PYG{n+nt}{head}\PYG{p}{\PYGZgt{}}
\PYG{p}{\PYGZlt{}}\PYG{n+nt}{title}\PYG{p}{\PYGZgt{}}Installing and Setting up Remote Lighting\PYG{p}{\PYGZlt{}}\PYG{p}{/}\PYG{n+nt}{title}\PYG{p}{\PYGZgt{}}
\PYG{p}{\PYGZlt{}}\PYG{p}{/}\PYG{n+nt}{head}\PYG{p}{\PYGZgt{}}
\PYG{p}{\PYGZlt{}}\PYG{n+nt}{body}\PYG{p}{\PYGZgt{}}
\PYG{p}{\PYGZlt{}}\PYG{n+nt}{article} \PYG{n+na}{id}\PYG{o}{=}\PYG{l+s}{\PYGZdq{}install\PYGZhy{}and\PYGZhy{}setup\PYGZdq{}}\PYG{p}{\PYGZgt{}}
\PYG{p}{\PYGZlt{}}\PYG{n+nt}{h1}\PYG{p}{\PYGZgt{}}Installing and Setting up Remote Lighting\PYG{p}{\PYGZlt{}}\PYG{p}{/}\PYG{n+nt}{h1}\PYG{p}{\PYGZgt{}}
\PYG{p}{\PYGZlt{}}\PYG{n+nt}{p}\PYG{p}{\PYGZgt{}}Installation of your lighting kit includes installing the light bulbs into
light fixtures, preparing the remote control, and programming lighting groups.\PYG{p}{\PYGZlt{}}\PYG{p}{/}\PYG{n+nt}{p}\PYG{p}{\PYGZgt{}}
\PYG{p}{\PYGZlt{}}\PYG{n+nt}{h2}\PYG{p}{\PYGZgt{}}Steps\PYG{p}{\PYGZlt{}}\PYG{p}{/}\PYG{n+nt}{h2}\PYG{p}{\PYGZgt{}}
\PYG{p}{\PYGZlt{}}\PYG{n+nt}{ul}\PYG{p}{\PYGZgt{}}
\PYG{p}{\PYGZlt{}}\PYG{n+nt}{li}\PYG{p}{\PYGZgt{}}
\PYG{p}{\PYGZlt{}}\PYG{n+nt}{p}\PYG{p}{\PYGZgt{}}Install light bulbs.\PYG{p}{\PYGZlt{}}\PYG{p}{/}\PYG{n+nt}{p}\PYG{p}{\PYGZgt{}}
\PYG{p}{\PYGZlt{}}\PYG{p}{/}\PYG{n+nt}{li}\PYG{p}{\PYGZgt{}}
\PYG{p}{\PYGZlt{}}\PYG{n+nt}{li}\PYG{p}{\PYGZgt{}}
\PYG{p}{\PYGZlt{}}\PYG{n+nt}{p}\PYG{p}{\PYGZgt{}}Prepare remote control.\PYG{p}{\PYGZlt{}}\PYG{p}{/}\PYG{n+nt}{p}\PYG{p}{\PYGZgt{}}
\PYG{p}{\PYGZlt{}}\PYG{p}{/}\PYG{n+nt}{li}\PYG{p}{\PYGZgt{}}
\PYG{p}{\PYGZlt{}}\PYG{n+nt}{li}\PYG{p}{\PYGZgt{}}
\PYG{p}{\PYGZlt{}}\PYG{n+nt}{p}\PYG{p}{\PYGZgt{}}Program lighting groups.\PYG{p}{\PYGZlt{}}\PYG{p}{/}\PYG{n+nt}{p}\PYG{p}{\PYGZgt{}}
\PYG{p}{\PYGZlt{}}\PYG{p}{/}\PYG{n+nt}{li}\PYG{p}{\PYGZgt{}}
\PYG{p}{\PYGZlt{}}\PYG{p}{/}\PYG{n+nt}{ul}\PYG{p}{\PYGZgt{}}
\PYG{p}{\PYGZlt{}}\PYG{n+nt}{section}\PYG{p}{\PYGZgt{}}
\PYG{p}{\PYGZlt{}}\PYG{n+nt}{h2}\PYG{p}{\PYGZgt{}}Example\PYG{p}{\PYGZlt{}}\PYG{p}{/}\PYG{n+nt}{h2}\PYG{p}{\PYGZgt{}}
\PYG{p}{\PYGZlt{}}\PYG{n+nt}{p}\PYG{p}{\PYGZgt{}}The following video demonstrates a recommended installation:\PYG{p}{\PYGZlt{}}\PYG{p}{/}\PYG{n+nt}{p}\PYG{p}{\PYGZgt{}}
\PYG{p}{\PYGZlt{}}\PYG{n+nt}{video} \PYG{n+na}{src}\PYG{o}{=}\PYG{l+s}{\PYGZdq{}remote.mp4\PYGZdq{}} \PYG{n+na}{controls} \PYG{n+na}{poster}\PYG{o}{=}\PYG{l+s}{\PYGZdq{}remote.png\PYGZdq{}}\PYG{p}{\PYGZgt{}}\PYG{p}{\PYGZlt{}}\PYG{p}{/}\PYG{n+nt}{video}\PYG{p}{\PYGZgt{}}
\PYG{p}{\PYGZlt{}}\PYG{n+nt}{p} \PYG{n+na}{data\PYGZhy{}conref}\PYG{o}{=}\PYG{l+s}{\PYGZdq{}bulbs\PYGZhy{}to\PYGZhy{}groups.dita\PYGZsh{}bulbs\PYGZhy{}to\PYGZhy{}groups/assign\PYGZhy{}disclaimer\PYGZdq{}}\PYG{p}{\PYGZgt{}}\PYG{p}{\PYGZlt{}}\PYG{p}{/}\PYG{n+nt}{p}\PYG{p}{\PYGZgt{}}
\PYG{p}{\PYGZlt{}}\PYG{p}{/}\PYG{n+nt}{section}\PYG{p}{\PYGZgt{}}
\PYG{p}{\PYGZlt{}}\PYG{p}{/}\PYG{n+nt}{article}\PYG{p}{\PYGZgt{}}
\PYG{p}{\PYGZlt{}}\PYG{p}{/}\PYG{n+nt}{body}\PYG{p}{\PYGZgt{}}
\PYG{p}{\PYGZlt{}}\PYG{p}{/}\PYG{n+nt}{html}\PYG{p}{\PYGZgt{}}
\end{sphinxVerbatim}


\section{MDITA}
\label{\detokenize{chap4:mdita}}
MDITA是基于Markdown的LwDITA,包含两种类型:
\begin{enumerate}
\sphinxsetlistlabels{\arabic}{enumi}{enumii}{}{.}%
\item {} 
\sphinxstylestrong{核心类型(core profile):}
对应GitHub Flavored Markdown(GFM)

\item {} 
\sphinxstylestrong{扩展类型(extended profile):}
针对于一些只在特定Markdown格式才中出现的功能,提供更加友好的DITA体验。

\end{enumerate}

\sphinxstylestrong{面向对象:}
由于MDITA分为两种类型,它的++潜在用户++也分为两类:

核心类型MDITA的用户:
\begin{itemize}
\item {} 
不想使用XML编辑器,但要为基于DITA的产品文档贡献内容的软件开发人员

\item {} 
负责API文档应用的开发或者技术写作者,需要将内容与其他技术文档应用共享

\item {} 
想要使用移动端(但不支持XML编辑器)编辑的写作者

\item {} 
希望内容后期能快速地转换为结构化的内容

\item {} 
希望利用DITA重用和发布机制,但不依赖于XML标签的写作者

\end{itemize}

扩展类型MDITA的用户:
\begin{itemize}
\item {} 
接收文档开发人员Markdown格式文件的内容管理人员

\item {} 
想将Markdown文件包含进DITA或者XDITA话题集合中的技术写作人员

\item {} 
想要使用移动端(但不支持XML编辑器)编辑的写作者

\sphinxstylestrong{举例}

\end{itemize}

MDITA \sphinxstyleemphasis{core profile}

MDITA核心类型中包含Markdown格式本身包含的信息架构,允许用户使用熟悉的写作方式编辑,例如:• Title;Paragraph;Section title;Ordered list;Unordered list;Link;Image;Preformatted text;Italics;Bold;Table;Code block

\begin{sphinxVerbatim}[commandchars=\\\{\}]
\PYG{g+gh}{\PYGZsh{}} Installing and Setting up Remote Lighting
Installation of your lighting kit includes installing the light bulbs into light
fixtures, preparing the remote control, and programming lighting groups.
\PYG{g+gu}{\PYGZsh{}\PYGZsh{}} Suggested Steps
\PYG{k}{1.} Install light bulbs.
\PYG{k}{2.} Prepare remote control.
\PYG{k}{3.} Program lighting groups.
\PYG{g+gu}{\PYGZsh{}\PYGZsh{}} Example
![\PYG{n+nt}{Remote installation example}](\PYG{n+na}{remote.png})
\end{sphinxVerbatim}

\sphinxstylestrong{使用LwDITA跨格式创建内容}
LwDITA支持跨格式的内容共享。作者可以同时使用XDITA,HDITA,MDITA,或DITA 1.3标准版进行创作,然后统一集合发布。除MDITA 核心类型外,用户可以在其他所有格式中使用content references和key reference。
跨格式内容共享的能力可以用来:
\begin{itemize}
\item {} 
使用XDITA地图链接XDITA,HDITA,MDITA,或DITA 1.3标准版格式的话题

\item {} 
使用XDITA话题重用MDITA话题

\item {} 
使用HDITA话题重用MDITA话题

\item {} 
使用MDITA扩展类型话题重用HDITA话题

\end{itemize}


\chapter{LwDITA工具}
\label{\detokenize{chap5:lwdita}}\label{\detokenize{chap5::doc}}
OASIS官方推荐用户可以使用以下(但不限于)的工具创建LwDITA内容:
\begin{itemize}
\item {} 
DITA Open Toolkit
\begin{itemize}
\item {} 
2.5.1及以上版本:可以创建XDITA内容

\item {} 
3.2及以上版本:可以创建接XDITA,HDITA,MDITA内容

\end{itemize}

\item {} 
Oxygen
\begin{itemize}
\item {} 
XML Editor/XML Author Author:提供XDITA、MDITA话题创建、编辑、发布的辅助。提供可以轻松创建LwDITA地图的文件模板。提供具有行间提示、表单控制等功能的XDITA编辑器。提供支持DITA预览的MDITA编辑器。

\item {} 
Web Author:一个基于浏览器的编辑器,支持XDITA话题,大致功能与XML Editor/XML Author桌面版本相同。

\end{itemize}

\item {} 
Adobe FrameMaker(2019及以上版本)在DITA1.1,1.2,1.3版本支持以外,还支持创建、编辑、发布XDITA话题。

\item {} 
Adobe Experience Manager XML Documentation(3.3及以上版本):支持创建、编辑XDITA话题。支持预览、翻译、发布XDITA话题、地图。同时支持DITA标准版本和XDITA。

\item {} 
XMetaL Author Enterprise (14及以上版本)支持LwDITA XML话题和地图的编辑。

\end{itemize}



\renewcommand{\indexname}{索引}
\printindex
\end{document}